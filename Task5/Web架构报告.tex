\documentclass[]{article}
\usepackage[noindent,UTF8]{ctex}

%opening
\title{Task5-Web应用架构报告}
\author{杨瑞,李静旭,耿豪}

\begin{document}

\maketitle

\section{概述}
我们采用的是We架构中的层次结构,将应用根据逻辑划分成不同的部分,然后单独考虑并单独编码实现,每一部分实现不同的功能,并可能分布在不同的主机,不同的操作系统上,通常以层( layer)来描述划分的领域。如图 \ref{fig:graph1} 所示
\begin{figure}[h]
	\centering
	\includegraphics[width=3in,height=3in]{layer.png}
	\caption{Web应用的三层架构}
	\label{fig:graph1}
\end{figure}

\section{展示层}
展示层就是用户和系统交互的层面,所以在我们的这个系统中,展示层所展现的就是为用户提供注册和登陆界面。如果用户用户还没有账号,那就先注册再登录;如果有账号则直接登陆。

\subsection{注册登陆界面}
注册界面如图 \ref{fig:graph2} 所示

\begin{figure}[h]
	\centering
	\includegraphics[width=3in,height=3in]{signup.png}
	\caption{注册界面}
	\label{fig:graph2}
\end{figure}

登陆界面如图 \ref{fig:graph3} 所示

\begin{figure}[h]
	\centering
	\includegraphics[width=3in,height=3in]{login.png}
	\caption{登陆界面}
	\label{fig:graph3}
\end{figure}


\section{业务逻辑层}
我们这个系统业务逻辑层很简单,就是登陆进去之后写微博内容,然后将其内容保存,另外其他用户包括自己也可以点赞。如图 \ref{fig:graph4} 所示

\begin{figure}[h]
	\centering
	\includegraphics[width=3in,height=3in]{content.png}
	\caption{内容页面}
	\label{fig:graph4}
\end{figure}

不过上面这个界面和功能实在是太low了,所以目前打算在该页面中嵌入富文本编辑器,这样发送内容的时候就不仅仅发文字了,还可以修该文字的大小和颜色,还可以发表情,图片,代码等丰富的功能。如图 \ref{fig:graph5} 所示是百度开发的UEeditor测试示例。

\begin{figure}[h]
	\centering
	\includegraphics[width=3in,height=3in]{UEeditor.png}
	\caption{UEedito预览}
	\label{fig:graph5}
\end{figure}


\section{数据层}
我们使用了MySQL数据库来保存数据,总共创建了四张表,其中四张表结构如图\ref{fig:graph6}所示。

\begin{figure}[h]
	\centering
	\includegraphics[width=3in,height=3in]{database.png}
	\caption{表结构}
	\label{fig:graph6}
\end{figure}

\end{document}
